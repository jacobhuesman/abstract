\documentclass{article}
%\documentclass{scrartcl}
\usepackage{scrextend}
\usepackage{enumitem}
\usepackage{parskip}
\usepackage{amssymb}
\usepackage{array}   % for \newcolumntype macro
\usepackage{amsthm}
\usepackage{chngcntr}


\counterwithin*{equation}{section}
\counterwithin*{equation}{subsection}

\newcolumntype{L}{>{$}c<{$}} % Define mathtype column

\begin{document}



\centerline{\sc \large Abstract Mathematics Homework 3}
\vspace{.5pc}
\centerline{\sc Jacob Huesman, 03 Feb 2016}

\vspace{2pc}


\section{Book Problems}
\subsection{Problem 3.2}
\renewcommand{\labelenumi}{(\alph{enumi})}
\begin{enumerate}[noitemsep, nolistsep]
	\item Let x be an integer. Prove that if x is odd, then $x^2$ is odd. Make sure you state your assumption as the first line and your conclusion as the last line.	
	\item State the contrapositive of what you just proved.
	\item Combining the result of part (a) with Theorem 3.3 gives a stronger result. Say precisely what that result is.
\end{enumerate}
	
\textit{Proof.}
\begin{addmargin}[0.75cm]{0cm}
	Since x is odd, we know that for some integer $n$, $x=2n+1$.
	
	We can then state the following:
	\begin{equation}
		x^2 = (2n+1)^2 = 4n^2+4n+1 
		= 2(2n^2+2n)+1
	\end{equation}
	
	So $x^2 = 2(2n^2 + 2n) + 1$, where $2n^2 + 2n$ is an integer k.
	
	Therefore $x^2=2k+1$, which matches the definition of an odd number.
	
	We can then conclude that if x is odd, then $x^2$ is odd.
	
	The contrapositive of this statement is, if $x^2$ is even, then x is even.

	\qed
\end{addmargin}

DO C


\pagebreak
\subsection{Problem 3.3}
For each of the following, write out the contrapositive and the converse of the sentence.
\begin{enumerate}[noitemsep, nolistsep]
	\item If you are the President of the United States, then you live in a white house.
	\item If you are going to bake a souffl\'e, then you need eggs.
	\item If x is a real number, then x is an integer.
	\item If x is a real number, then $x^2<0$.
\end{enumerate}

\subsubsection{If you are the President of the United States, then you live in a white house.}
Contrapositive: 
If you don't live in a white house, you are not the President of the United States.

Converse: 
If you live in a white house, you are the President of the United States.

\subsubsection{If you are going to bake a souffl\'e, then you need eggs.}
Contrapositive: 
If you don't need eggs, then you are not going to bake a souffl\'e.

Converse: 
If you need eggs, you are going to bake a souffl\'e.

\subsubsection{If x is a real number, then x is an integer.}
Contrapositive: 
If x is not an integer, then x is not a real number.

Converse: 
If x is an integer, then x is a real number.

\subsubsection{If x is a real number, then \boldmath $x^2<0$}
Contrapositive: 
If x is not an integer, then x is not a real number.

Converse: 
If x is an integer, then x is a real number.


\pagebreak
\subsection{Problem 3.14}
Let $n$ be an integer. Prove that if $3n$ is odd, then $n$ is odd.

\textit{Proof.}
\begin{addmargin}[0.75cm]{0cm}

The contrapositive of the above statement is, if n is even, then 3n is even.

An even number is defined as $x=2m$, where $m \epsilon \mathbb{Z}$.

So $3n=3(2m)=2(3m)=2q$ where $q \epsilon \mathbb{Z}$.

Knowing $q \epsilon \mathbb{Z}$, we find that if $n$ is even, then $3n$ is even.

The above statement being the contrapositive of what we were trying to prove, we conclude
that if $3n$ is odd, then $n$ is odd.

\qed
\end{addmargin}


\subsection{Problem 3.15}
Let $x$ be a natural number. Prove that if $x$ is odd, then $\sqrt{2x}$ is not an integer.

\textit{Proof.}
\begin{addmargin}[0.75cm]{0cm}
	
	The contrapositive of the above statement is, if $\sqrt{2x}$ is an integer, then $x$ is even.
	
	If x is odd and $x \epsilon \mathbb{N}$, $x = 2n+1$ for $n \epsilon \mathbb{N}$
	
	Substituting this into $\sqrt{2x}$, you get $\sqrt{2(2n+1)}$, where $(2n+1)$ is clearly an integer.
	
	An integer multiplied by an integer will always return an integer, so $2(2n + 1)$ is also an integer.
	
	\begin{equation}
		\sqrt{2(2n+1)} = \sqrt{4(k+1/2)} = 2 \sqrt{k+1/2}
	\end{equation}
	
	From the above equation we conclude that $\sqrt{k+1/2}$ is not an integer and therefore 
	if $x$ is odd, then $\sqrt{2x}$ is not an integer.
	
	\qed
\end{addmargin}

\pagebreak
\subsection{Problem 3.16}
Let $x$ and $y$ be real numbers. Show that if $x \neq y$ and $x,y \geq 0$, then $x^2 \neq y^2$.

\textit{Proof.}
\begin{addmargin}[0.75cm]{0cm}
	We start by taking the contrapositive of the above statement.
	
	If $x^2 = y^2$, then $x = y$ and $x,y \geq 0$.
	
	For $a$ to equal $b$, $\frac{a}{b}$ must equal 1.
	
	If follows that $\frac{x^2}{y^2}=(\frac{x}{y})^2=1$.
	
	If we take the square root of both sides we get $\frac{x}{y}=1$.
	
	Therefore if $x^2=y^2$, then $x=y$.
	
	We conclude via contrapositive that if $x \neq y$ and $x,y \geq 0$, then $x^2 \neq y^2$.
	
	\qed
\end{addmargin}


\pagebreak
\section{Group Problem}


\end{document}