\documentclass{article}
%\documentclass{scrartcl}
\usepackage{scrextend}
\usepackage{enumitem}
\usepackage{parskip}
\usepackage{amssymb}
\usepackage{array}   % for \newcolumntype macro
\usepackage{amsthm}
\usepackage{chngcntr}
\usepackage{mathtools}


\counterwithin*{equation}{section}
\counterwithin*{equation}{subsection}

\newcolumntype{L}{>{$}c<{$}} % Define mathtype column

\begin{document}



\centerline{\sc \large Exam 1 Study Guide}
\vspace{.5pc}
\centerline{\sc Jacob Huesman, 07 Feb 2016}

\vspace{2pc}

\underline{Theorem}
\begin{addmargin}[0.75cm]{0cm}
	If $n$ is an integer, then $n^2 + 3n + 2$ is an even integer.
\end{addmargin}

\underline{Lemma}
\renewcommand{\labelenumi}{(\alph{enumi})}
\begin{enumerate}[leftmargin=1.35cm]
	\item An even number multiplied by an odd number is even.
	\item A given integer that is even or odd, remains even or odd when summed with an even integer.
\end{enumerate}

\underline{Proof}
\begin{addmargin}[0.75cm]{0cm}
	We start by factoring $n^2 + 3n +2$ to get $n(n+3)+2$.
	
	Suppose $n$ is even, then $(n+3)$ must be odd, as 
	\begin{equation}
		(n+3)=(2k+3):k\in\mathbb{Z}=2k+1
	\end{equation}
	
	For $n \in 2\mathbb{Z}$ the expression $n(n+3)+2$ is then an even number by lemma a and b.
	
	Now suppose $n$ is odd, then $(n+3)$ must be even as 
	\begin{equation}
		(n+3)=(2k+1+3): k \in \mathbb{Z}=2k
	\end{equation}
	
	For $n \in 2\mathbb{Z}+1$ the expression $n(n+3)+2$ is then an even number by lemma a and b.
	
	We have proven that the expression is even when $n$ is even and $n$ is odd.
	
	The union of the two sets is the integer set, so we conclude the following.
	
	If $n$ is an integer, then $n^2 + 3n + 2$ is an even integer.
			
	\qed
\end{addmargin}


\end{document}