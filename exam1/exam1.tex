\documentclass{article}
%\documentclass{scrartcl}
\usepackage{scrextend}
\usepackage{enumitem}
\usepackage{parskip}
\usepackage{amssymb}
\usepackage{array}   % for \newcolumntype macro
\usepackage{amsthm}
\usepackage{chngcntr}
\usepackage{mathtools}


\counterwithin*{equation}{section}
\counterwithin*{equation}{subsection}

\newcolumntype{L}{>{$}c<{$}} % Define mathtype column

\begin{document}



\centerline{\sc \large Exam 1 Study Guide}
\vspace{.5pc}
\centerline{\sc Jacob Huesman, 07 Feb 2016}

\vspace{2pc}

\section{Chapter 5. Proof Techniques}
\subsection{Problem 5.3}
\underline{Theorem}
\begin{addmargin}[0.75cm]{0cm}
	If $n$ is an integer, then $n^2 + 3n + 2$ is an even integer.
\end{addmargin}

\underline{Lemma}
\renewcommand{\labelenumi}{(\alph{enumi})}
\begin{enumerate}[leftmargin=1.35cm]
	\item An even number multiplied by an odd number is even.
	\item A given integer that is even or odd, remains even or odd when summed with an even integer.
\end{enumerate}

\underline{Proof}
\begin{addmargin}[0.75cm]{0cm}
	We start by factoring $n^2 + 3n +2$ to get $n(n+3)+2$.
	
	Suppose $n$ is even, then $(n+3)$ must be odd, as 
	\begin{equation}
		(n+3)=(2k+3):k\in\mathbb{Z}=2k+1
	\end{equation}
	
	For $n \in 2\mathbb{Z}$ the expression $n(n+3)+2$ is then an even number by lemma a and b.
	
	Now suppose $n$ is odd, then $(n+3)$ must be even as 
	\begin{equation}
		(n+3)=(2k+1+3): k \in \mathbb{Z}=2k
	\end{equation}
	
	For $n \in 2\mathbb{Z}+1$ the expression $n(n+3)+2$ is then an even number by lemma a and b.
	
	We have proven that the expression is even when $n$ is even and $n$ is odd.
	
	The union of the two sets is the integer set, so we conclude through proof by cases the following.
	
	If $n$ is an integer, then $n^2 + 3n + 2$ is an even integer.
			
	\qed
\end{addmargin}


\pagebreak


\subsection{Problem 5.13}
\underline{Theorem}
\begin{addmargin}[0.75cm]{0cm}
	$sin^2(x) \leq |sin(x)|$ for all $x \in \mathbb{R}$
\end{addmargin}

\underline{Definition}
\renewcommand{\labelenumi}{(\alph{enumi})}
\begin{enumerate}[leftmargin=1.35cm]
	\item $sin(x) \in \{-1 \leq y \leq 1 : y \in \mathbb{R}\}$
\end{enumerate}

\underline{Lemma}
\begin{enumerate}[leftmargin=1.35cm]
	\item $x^2 \leq |x|$ for $-1 \leq x \leq 1$ where $x \in \mathbb{R}$
\end{enumerate}

\underline{Proof}
\begin{addmargin}[0.75cm]{0cm}
	We start by proving Lemma a.
	
	Note that the range of $x^2$ and $|x|$ for $-1 \leq x \leq 1$ is limited to $[0,1]$.
	
	A real number $x$ where $0 < x \leq 1$ can be represented by $x = \frac{1}{a}, a \in \{\mathbb{N}>0\}$
	
	Observe that $x^2 = (\frac{1}{a})^2 = \frac{1}{a^2}$.
	
	Since $a^2 \geq a$ for any integer greater than 0, $(\frac{1}{a})^2 \leq \frac{1}{a}$.
	
	The above is equivalent to saying $x^2 \leq x$ for $x \in \mathbb{R}$ where $0 < x \leq 1$.
	
	This leaves the case where x = 0. 
	
	Well if $x = 0$, then $x^2 = 0$ as well.
	
	So $x^2 \leq |x|$ for $-1 \leq x \leq 1$ where $x \in \mathbb{R}$.
	
	\qed
	
	Note that the range of $sin^2(x)$ and $|sin(x)|$ is $[0,1]$.
	
	We can then conclude by Lemma a the following.
	\begin{equation}
		sin^2(x) \leq |sin(x)| \text{ for all } x \in \mathbb{R}
	\end{equation}	
	
	\qed
\end{addmargin}


\pagebreak

\underline{Theorem}
\begin{addmargin}[0.75cm]{0cm}
	TODO
\end{addmargin}

\underline{Lemma}
\renewcommand{\labelenumi}{(\alph{enumi})}
\begin{enumerate}[leftmargin=1.35cm]
	\item TODO
	\item TODO
\end{enumerate}

\underline{Proof}
\begin{addmargin}[0.75cm]{0cm}
	TODO
	
	\qed
\end{addmargin}


\pagebreak

\underline{Theorem}
\begin{addmargin}[0.75cm]{0cm}
	TODO
\end{addmargin}

\underline{Lemma}
\renewcommand{\labelenumi}{(\alph{enumi})}
\begin{enumerate}[leftmargin=1.35cm]
	\item TODO
	\item TODO
\end{enumerate}

\underline{Proof}
\begin{addmargin}[0.75cm]{0cm}
	TODO
	
	\qed
\end{addmargin}

\end{document}