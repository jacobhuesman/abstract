\documentclass{article}
%\documentclass{scrartcl}
\usepackage{scrextend}
\usepackage{enumitem}
\usepackage{parskip}
\usepackage{amssymb}
\usepackage{array}   % for \newcolumntype macro
\usepackage{amsthm}
\usepackage{chngcntr}
\usepackage{mathtools}


\counterwithin*{equation}{section}
\counterwithin*{equation}{subsection}

\newcolumntype{L}{>{$}c<{$}} % Define mathtype column

\begin{document}



\centerline{\sc \large Abstract Mathematics Homework 3}
\vspace{.5pc}
\centerline{\sc Jacob Huesman, 03 Feb 2016}

\vspace{2pc}


\section{The How, When, and Why of Mathematics}
\begin{enumerate}[noitemsep, nolistsep]
	\item \textbf{Polya's method}
		\subitem a. Understanding the problem
		\subitem b. Devising a plan
		\subitem c. Carrying out the plan
		\subitem d. Looking back
\end{enumerate} 
	

\section{Logically Speaking}
\textbf{Theorem 2.7.} 
Two statement forms $P$ and $Q$ are equivalent if and only if they have the same truth table.

\textbf{Theorem 2.9.}
Let P and Q denote statement forms. The following are tautologies:
\begin{enumerate}[noitemsep, nolistsep]
	\item \textbf{DeMorgan's laws}
	\subitem  $\lnot(P \lor Q) \leftrightarrow (\lnot P \land \lnot Q)$
	\subitem  $\lnot(P \land Q) \leftrightarrow (\lnot P \lor \lnot Q)$
	\item \textbf{Implication and its negation}
	\subitem $(P \rightarrow Q) \leftrightarrow (\lnot P \lor Q)$
	\subitem $\lnot(P \rightarrow Q) \leftrightarrow (P \lor \lnot Q)$
	\item \textbf{Double negation}
	\subitem $\lnot(\lnot P) \leftrightarrow P$
\end{enumerate} 


\section{Introducing the Contrapositive and Converse}
\subsection{Definitions}
\textbf{Definition 3.1.} An \textbf{integer} $x$ \textbf{is odd} if there is an integer $n$ such that $x = 2n +1$.

\textbf{Definition 3.2.} An \textbf{integer} $x$ \textbf{is even} if there is an integer $n$ such that $x = 2n$

\textbf{Definition 3.3.} An integer $p$ is \textbf{prime} if $p > 1$ and $p$ cannot be written as a product of two positive integers, both different from $p$.


\subsection{Theorems}
\textbf{Theorem 3.1.} Let $P$, $Q$, and $R$ denote statement forms. Then the following are tautologies:
\begin{enumerate}[noitemsep, nolistsep]
	\item \textbf{Distributive property}
		\subitem  $(P \land (Q \lor R)) \leftrightarrow ((P \land Q) \lor (P \land R))$
		\subitem  $(P \lor (Q \land R)) \leftrightarrow ((P \lor Q) \land (P \lor R))$
	\item \textbf{Associative property}
		\subitem $(P \land (Q \land R)) \leftrightarrow ((P \land Q) \land R)$
		\subitem $(P \lor (Q \lor R)) \leftrightarrow ((P \lor Q) \lor R)$

	\item \textbf{Commutative property}
		\subitem  $(P \land Q) \leftrightarrow (Q \land P)$
		\subitem  $(P \lor Q) \leftrightarrow (Q \lor P)$
\end{enumerate} 

\textbf{Theorem 3.3.} Let $x$ be an integer. If $x^2$ is odd, then $x$ is odd.

\textbf{Theorem (Contrapositive of the statement of Theorem 3.3).} Let $x$ be an integer. If $x$ is even, then $x^2$ is even.


\section{Set Notation and Quantifiers}
\subsection{Common Sets}
The natural numbers : $\mathbb{N} = \{0,1,2,3,...\}$

The integers : $\mathbb{Z} = \{...,-2,-1,0,1,2,...\}$

The rational numbers : $\mathbb{Q} = \{p/q : p,q \in \mathbb{Z}\ and\ q \neq 0 \}$

The real numbers : $\mathbb{R}$

The complex numbers : $\mathbb{C} = \{a + bi : i^2 = -1 \ and \ a,b \in \mathbb{R}\}$

If A is one of the sets $\mathbb{Z}$, $\mathbb{Q}$, or $\mathbb{R}$, then the set of the positive elements is denoted by $A^+ = \{x \in A : x > 0 \}$ and the set of the negative elements is denoted by $A^- = \{x \in A : x < 0 \}$. Thus we have defined $\mathbb{Z}^+$, $\mathbb{Z}^-$, $\mathbb{Q}^+$, $\mathbb{Q}^-$, $\mathbb{R}^+$, and $\mathbb{R}^-$

The plane $\mathbb{R}^2 = \{(x,y) : x,y \in \mathbb{R} \}$

For $n \in \mathbb{Z}^+$, Euclidean $n$-space 
	$\mathbb{R}^n = \{(x_{1},x_{2},...,x_{n}) : x_{j} \in \mathbb{R}$ for $j=1,2,...,n \}$.
	
\subsection{Symbols}
For all $\forall$

There exists $\exists$

\section{Proof Techniques}
\subsection{Definitions}
Three methods discussed in this chapter:
\begin{itemize}
	\item direct proof (just get started and keep going)
	\item proof by contradiction (show that the negation of the statement you wish to prove implies the impossible)
	\item proof in cases (which may be used when conditions dictate that different situations occur).
\end{itemize}

\textbf{Definition 5.1.} A nonzero integer $a$ \textbf{divides} an integer $b$ if there is an integer $n$ such that $b = an$. We write this as $a|b$.

\textbf{Definition 5.2.} For a real number x, the \textbf{absolute value} of $x$ is defined to be

\begin{center}
	\begin{math}
		|x| =
			\begin{cases*}
				 x & if $x \geq 0$ \\
				-x & if $x < 0$
			\end{cases*}
	\end{math}
\end{center}

\subsection{Theorems}
\textbf{Theorem 5.1.} If $a$, $b$ and $c$ are integers such that $a$ divides $b$ and $a$ divides $c$, then $a$ divides $b$ + $c$.

\textbf{Theorem 5.2.} The number $\sqrt{2}$ is not rational.

\textbf{Theorem 5.3.} Let $x$ and $y$ be real numbers. Then $|xy| = |x||y|$.


\section{Sets}
\subsection{Definitions}
\textbf{Definition 6.1.} The \textbf{empty set}, denoted $\emptyset$ is the set with no elements.

\textbf{Definition 6.2.} The set $A$ is a \textbf{subset} of the set $B$ or, equivalently, $A$ is \textbf{contained} in $B$, if every element of A is an element of $B$. We write $A \subseteq B$ to indicate that $A$ is a subset of $B$. 

\textbf{Definition 6.3.} The set $A$ is a \textbf{proper subset} of $B$ if $A \subseteq B$ and $A \neq B$, and we write $A \subset B$.

\textbf{Definition 6.4.} The set $A$ is \textbf{equal} to $B$, written $A = B$, if $A \subseteq B$ and $B \subseteq A$.

\textbf{Definition 6.5.} The \textbf{union} of the sets $A$ and $B$ is the set 
	$A \cup B = \{x : x \in A\ or\ x \in B \}$.
	
\textbf{Definition 6.6.} The \textbf{intersection} of the sets $A$ and $B$ is the set 
	$A \cap B = \{x : x \in A\ and\ x \in B \}$.
	
\textbf{Definition 6.7.} Two sets $A$ and $B$ are \textbf{disjoint} if $A \cap B = \emptyset$.

\textbf{Definition 6.8.} The \textbf{set difference} of set $B$ in set $A$ is the set 
	$A \setminus B = \{x \in A : x \notin B \}$.
	
\textbf{Definition 6.9.} If the set X is the universe and A is a subset of X, the  \textbf{complement} of A is the set $A^c = X \setminus A$.

\subsection{Theorem 6.11.} Let $A$ be a set. Then $\emptyset \subseteq A$. 




\end{document}