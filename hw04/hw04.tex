\documentclass{article}
%\documentclass{scrartcl}
\usepackage{scrextend}
\usepackage{enumitem}
\usepackage{parskip}
\usepackage{amssymb}
\usepackage{array}   % for \newcolumntype macro
\usepackage{amsthm}
\usepackage{amsmath}
\usepackage{chngcntr}


\counterwithin*{equation}{section}
\counterwithin*{equation}{subsection}

\newcolumntype{L}{>{$}c<{$}} % Define mathtype column

\begin{document}



\centerline{\sc \large Abstract Mathematics Homework 4}
\vspace{.5pc}
\centerline{\sc Jacob Huesman, 22 Feb 2016}

\vspace{2pc}


\section{Book Problems}
\subsection{Theorem 7.4.9}
Let $X$ denote a set and A,B, and C denote subsets of X. Then
\begin{equation}
	(A \cup B) \cup C = A \cup (B \cup C)
\end{equation}
	
\textbf{\textit{Proof.}}

For sets $X$ and $Y$, $X \cup Y$ is defined a follows.
\begin{equation}
	\text{If } X \cup Y \text{, then } x \in X \text{ or } x \in Y
\end{equation}
So $A \cup B$ is the set where $x \in A$ or $x \in B$.

The union of $(A \cup B)$ with $C$ is then the set where $x \in A$, $x \in B$, or $x \in C$.

Now if $B \cup C$, then $x \in B$ or $x \in C$.

The union of $A \cup (B \cup C)$ is then the set where $x \in A$, $x \in B$, or $x \in C$.

Since both the set $(A \cup B) \cup C$ and $A \cup (B \cup C)$ share the same definition, we
can conclude that $(A \cup B) \cup C \subseteq A \cup (B \cup C)$, conversely that 
$A \cup (B \cup C) \subseteq (A \cup B) \cup C$, and as a result of the last two conclusions,
that $(A \cup B) \cup C = A \cup (B \cup C)$.

\qed


\pagebreak
\subsection{Theorem 7.4.15}
Let $X$ denote a set and A,B, and C denote subsets of X. Then
\begin{equation}
	X \setminus (A \cup B) = (X \setminus A) \cap (X \setminus B)
\end{equation}

\textbf{\textit{Proof.}}
	
Observe the following:
\begin{enumerate}
	\item If $x \in X \setminus (A \cup B)$, then $x \in X$ and $x \not\in (A \cup B)$.
	\item If $x \in (X \setminus A) \cap (X \setminus B)$, then $x \in X$, $x \not\in A$, and $x \not\in B$.
\end{enumerate}

We must first prove that $X \setminus (A \cup B) \subseteq (X \setminus A) \cap (X \setminus B)$.

Notice for the set where $x \in X \setminus (A \cup B)$, that $x \in X$, therefore what remains to show is that $x \not\in A$ and $x \not\in B$.

The set where $x \not\in A \cup B$ can be defined as the set  $\{x\ |\ \lnot(x \in A \lor x \in B)\}$.

By DeMorgan's Laws of Logic we can say that this is equivalent to the set 
$\{x\ |\ x \not\in A \land x \not\in B\}$

Hence $x \not\in A$, $x \not\in B$, and $X \setminus (A \cup B) \subseteq (X \setminus A) \cap (X \setminus B)$.

We now must show that $(X \setminus A) \cap (X \setminus B) \subseteq X \setminus (A \cup B)$.

The set $(X \setminus A) \cap (X \setminus B)$ can be defined as the set where $\{x\ |\ x \in X \land x \not\in A \land x \not\in B \}$.

By DeMorgan's Laws of Logic we can say this is equivalent to the set $\{x\ |\ x \in X \land \lnot(x \in A \lor x \in B) \}$, which is the set $x \in X \setminus (A \cup B)$.

We have found that $(X \setminus A) \cap (X \setminus B) \subseteq X \setminus (A \cup B)$ and conclude that:
\begin{equation}
	X \setminus (A \cup B) = (X \setminus A) \cap (X \setminus B)
\end{equation}

\qed



\pagebreak
\subsection{Theorem 7.4.17}
Let $X$ denote a set and A,B, and C denote subsets of X. Then
\begin{equation}
	A \setminus B = A \cap B^c
\end{equation}

\textbf{\textit{Proof.}}

Observe the following:
\begin{enumerate}
	\item $A^c = \{x\ |\ x \not\in B \}$
	\item $A \cap B = \{x\ |\ x \in A \land x \in B \}$
\end{enumerate}

Following the definitions observed above we can describe $A \cap B^c$ as follows:
\begin{equation}
	A \cap B^c = \{x\ |\ x \in A \land x \not\in B \}
\end{equation}
Well that's the definition of $A \setminus B$, so we conclude $A \setminus B = A \cap B^c$.

\qed



\pagebreak
\subsection{Theorem 7.4.19}
Let $X$ denote a set and A,B, and C denote subsets of X. Then
\begin{equation}
	(A \subseteq C) \land (B \subseteq C) \iff (A \cup B) \subseteq C
\end{equation}

\textbf{\textit{Proof.}}

Observe the following:
\begin{enumerate}
	\item $A \subseteq B$ indicates that every element of A is in B.
	\item $A \cup B = \{x\ |\ x \in A \lor x \in B \}$
\end{enumerate}

We start by proving $(A \subseteq C) \land (B \subseteq C) \rightarrow (A \cup B) \subseteq C$.

The statement form $(A \cup B) \subseteq C$ states that every element in the sets A,B is contained in C.

So the set A must be contained in C and the set B must be contained in C.

We have shown $(A \subseteq C) \land (B \subseteq C) \rightarrow (A \cup B) \subseteq C$.

Now we must prove $(A \cup B) \subseteq C \rightarrow (A \subseteq C) \land (B \subseteq C)$.

Let $A \subseteq C$, $B \subseteq C$, and $x \in (A \cup B)$. 

Then $x \in A$ or $x \in B$, which are both subsets of C.

This makes $(A \cup B)=\{x\ |\ x \in A \lor x \in B \}$ a subset of C.

We have shown $(A \cup B) \subseteq C \rightarrow (A \subseteq C) \land (B \subseteq C)$.

We conclude that $(A \subseteq C) \land (B \subseteq C) \iff (A \cup B) \subseteq C$

\qed


\pagebreak
\subsection{Problem 7.15}
Prove that $A^c \cup B^c = X$ if and only if $A$ and $B$ are disjoint.

\textbf{\textit{Proof.}}

Observe that $A$ and $B$ are disjoint if $A \cap B = \emptyset$. We assume X is our universe.

To prove that  ``$A^c \cup B^c = X$ if and only if $A$ and $B$ are disjoint", we must first prove that $A^c \cup B^c = X \rightarrow A \cap B = \emptyset$, and then prove $A \cap B = \emptyset \rightarrow A^c \cup B^c = X$.

Let $A \cap B = \emptyset$. When we take the complement of both sides of the statement we get
$(A \cap B)^c = \emptyset^c = A^c \cup B^c = X$. So $A^c \cup B^c = X \rightarrow A \cap B = \emptyset$.

Let $A^c \cup B^c = X$. When we take the complement of both sides of the statement we get
$(A^c \cup B^c)^c = X^c = A \cap B = \emptyset$. So $A \cap B = \emptyset \rightarrow A^c \cup B^c = X$.

We conclude that $A^c \cup B^c = X$ if and only if $A$ and $B$ are disjoint.


\qed

\end{document}
